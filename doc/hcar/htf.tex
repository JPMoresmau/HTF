\documentclass{scrreprt}
\usepackage{paralist}
\usepackage{graphicx}
\usepackage[final]{hcar}

\begin{document}

\begin{hcarentry}{HTF: a test framework for Haskell}
\report{Stefan Wehr}
\status{beta, active development}
\makeheader
\label{htf}

The Haskell Test Framework (\emph{HTF} for short) lets you define unit
tests, QuickCheck properties, and black box tests in an easy and
convenient way. The HTF uses a custom preprocessor that collects test
definitions automatically. Furthermore, the preprocessor allows the HTF to
report failing test cases with exact file name and line number
information.

Initially created in 2005, HTF wasn't actively developed for almost five
years. Development resumed in 2010, adding many improvements to the code
base.

\FurtherReading
\begin{itemize}
\item Hackage entry: \url{http://hackage.haskell.org/package/HTF}
\item Tutorial: \url{http://www.factisresearch.com/2010/03/htf/}
\end{itemize}

\end{hcarentry}

\end{document}
